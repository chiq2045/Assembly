\documentclass[12pt]{article}

\usepackage{answers}
\usepackage{setspace}
\usepackage{minted}
\usepackage{tcolorbox}
\usepackage{etoolbox}
\usepackage{graphicx}
\usepackage{enumitem}
\usepackage{multicol}
\usepackage{mathrsfs}
\usepackage[margin=1in]{geometry} 
\usepackage{amsmath,amsthm,amssymb}
 
\newcommand{\N}{\mathbb{N}}
\newcommand{\Z}{\mathbb{Z}}
\newcommand{\C}{\mathbb{C}}
\newcommand{\R}{\mathbb{R}}

\DeclareMathOperator{\sech}{sech}
\DeclareMathOperator{\csch}{csch}
 
\newenvironment{theorem}[2][Theorem]{\begin{trivlist}
\item[\hskip \labelsep {\bfseries #1}\hskip \labelsep {\bfseries #2.}]}{\end{trivlist}}
\newenvironment{definition}[2][Definition]{\begin{trivlist}
\item[\hskip \labelsep {\bfseries #1}\hskip \labelsep {\bfseries #2.}]}{\end{trivlist}}
\newenvironment{proposition}[2][Proposition]{\begin{trivlist}
\item[\hskip \labelsep {\bfseries #1}\hskip \labelsep {\bfseries #2.}]}{\end{trivlist}}
\newenvironment{lemma}[2][Lemma]{\begin{trivlist}
\item[\hskip \labelsep {\bfseries #1}\hskip \labelsep {\bfseries #2.}]}{\end{trivlist}}
\newenvironment{exercise}[2][Exercise]{\begin{trivlist}
\item[\hskip \labelsep {\bfseries #1}\hskip \labelsep {\bfseries #2.}]}{\end{trivlist}}
\newenvironment{solution}[2][Solution]{\begin{trivlist}
\item[\hskip \labelsep {\bfseries #1}]}{\end{trivlist}}
\newenvironment{problem}[2][Problem]{\begin{trivlist}
\item[\hskip \labelsep {\bfseries #1}\hskip \labelsep {\bfseries #2.}]}{\end{trivlist}}
\newenvironment{question}[2][Question]{\begin{trivlist}
\item[\hskip \labelsep {\bfseries #1}\hskip \labelsep {\bfseries #2.}]}{\end{trivlist}}
\newenvironment{corollary}[2][Corollary]{\begin{trivlist}
\item[\hskip \labelsep {\bfseries #1}\hskip \labelsep {\bfseries #2.}]}{\end{trivlist}}
 
\BeforeBeginEnvironment{minted}{\dontdofcolorbox}
\def\dontdofcolorbox{\renewcommand\fcolorbox[4][]{##4}}
\BeforeBeginEnvironment{minted}{\begin{tcolorbox}}%
\AfterEndEnvironment{minted}{\end{tcolorbox}}%

\begin{document}
 
% --------------------------------------------------------------
%                         Start here
% --------------------------------------------------------------
 
\title{Chapter 4 Homework}%replace with the appropriate homework number
\author{Charles Oroko} %if necessary, replace with your course title
\date{8 November 2018}
\maketitle
%Below is an example of the problem environment

% --------------------------------------------------------------

\begin{problem}{4.1}
    If $r0$ initially contains $1$, what will it contain after the third instruction in the sequence below?
    \begin{minted}{gas}
        add     r0, r0, #1
        mov     r1, r0
        add     r0, r1, r0, lsl #1
    \end{minted}

\end{problem}

$r0=6$

% --------------------------------------------------------------

\begin{problem}{4.2}
    What will r0 and r1 contain after each of the following instructions? Give your answers in base 10.
    \begin{minted}{gas}
        mov     r0, #1
        mov     r1, #0x20
        orr     r1, r1, r0
        lsl     r1, #0x2
        orr     r1, r1, r0
        eor     r0, r0, r1
        lsr     r1, r0, #3
    \end{minted}

    \begin{table}
        \begin{tabular}{cc}
            r0 & r1 \\
            \hline
            1 & 0\\
            1 & 32\\
            1 & 33\\
            1 & 132\\
            1 & 133\\
            132 & 133\\
            132 & 1056
        \end{tabular}
    \end{table}
\end{problem}

\pagebreak
% --------------------------------------------------------------

\begin{problem}{4.3}
    What is the difference between $lsr$ and $asr$?
\end{problem}

The $lsr$ and $asr$ operations do similar things. They both shifts each bit $n$ bits to the right, losing the least significant $n$ bits.

With the $lsr$ operation, zero is shifted into the $n$ most significant bits.
However, with the $asr$ operation, the $n$ most significant bits become copies of the sign bit (bit $31$).

% --------------------------------------------------------------

\begin{problem}{4.4}
    Write the ARM assembly code to load the numbers stored in $num1$ and $num2$, add them together, and store the result in $numsum$. Use only $r0$ and $r1$.
\end{problem}

\begin{minted}{gas}
    ldr     r0, =num1
    ldr     r1, =num2
    ldr     r0, [r0]
    ldr     r1, [r1]
    add     r0, r0, r1
    
    ldr     r1, =numsum
    str     r0, [r1]
\end{minted}

% --------------------------------------------------------------

\begin{problem}{4.5}
    Given the following variable definitions:
    \begin{minted}{gas}
        num1:   .word x
        num2:   .word y
    \end{minted}
    where you do not know the values of x and y, write a short sequence of ARM assembly instructions to load the two numbers, compare them, and move the largest number into register r0.
\end{problem}

\begin{minted}{gas}
    ldr     r1, =num1
    ldr     r2, =num2
    ldr     r0, [r1]
    ldr     r2, [r2]
    cmp     r0, r2
    movle   r0, r2
\end{minted}

\pagebreak
% --------------------------------------------------------------

\begin{problem}{4.6}
    Assuming that a is stored in register r0 and b is stored in register r1, show the ARM assembly code that is equivalent to the following C code.
    \begin{minted}{c}
        if ( a & 1 )
        a = -a;
        else
        b = b+7;
    \end{minted}

\end{problem}

\begin{minted}{gas}
    tst     r0, #1      @if(a&1)
    rsbne   r0, r0, #0  @a=-a
    addeq   r1, r1, #7  @else b+=7
\end{minted}

\end{document}
