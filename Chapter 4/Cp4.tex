\documentclass[12pt]{article}

\usepackage{answers}
\usepackage{setspace}
\usepackage{minted}
\usepackage{tcolorbox}
\usepackage{etoolbox}
\usepackage{graphicx}
\usepackage{enumitem}
\usepackage{multicol}
\usepackage{mathrsfs}
\usepackage[margin=1in]{geometry} 
\usepackage{amsmath,amsthm,amssymb}
 
\newcommand{\N}{\mathbb{N}}
\newcommand{\Z}{\mathbb{Z}}
\newcommand{\C}{\mathbb{C}}
\newcommand{\R}{\mathbb{R}}

\DeclareMathOperator{\sech}{sech}
\DeclareMathOperator{\csch}{csch}
 
\newenvironment{theorem}[2][Theorem]{\begin{trivlist}
\item[\hskip \labelsep {\bfseries #1}\hskip \labelsep {\bfseries #2.}]}{\end{trivlist}}
\newenvironment{definition}[2][Definition]{\begin{trivlist}
\item[\hskip \labelsep {\bfseries #1}\hskip \labelsep {\bfseries #2.}]}{\end{trivlist}}
\newenvironment{proposition}[2][Proposition]{\begin{trivlist}
\item[\hskip \labelsep {\bfseries #1}\hskip \labelsep {\bfseries #2.}]}{\end{trivlist}}
\newenvironment{lemma}[2][Lemma]{\begin{trivlist}
\item[\hskip \labelsep {\bfseries #1}\hskip \labelsep {\bfseries #2.}]}{\end{trivlist}}
\newenvironment{exercise}[2][Exercise]{\begin{trivlist}
\item[\hskip \labelsep {\bfseries #1}\hskip \labelsep {\bfseries #2.}]}{\end{trivlist}}
\newenvironment{solution}[2][Solution]{\begin{trivlist}
\item[\hskip \labelsep {\bfseries #1}]}{\end{trivlist}}
\newenvironment{problem}[2][Problem]{\begin{trivlist}
\item[\hskip \labelsep {\bfseries #1}\hskip \labelsep {\bfseries #2.}]}{\end{trivlist}}
\newenvironment{question}[2][Question]{\begin{trivlist}
\item[\hskip \labelsep {\bfseries #1}\hskip \labelsep {\bfseries #2.}]}{\end{trivlist}}
\newenvironment{corollary}[2][Corollary]{\begin{trivlist}
\item[\hskip \labelsep {\bfseries #1}\hskip \labelsep {\bfseries #2.}]}{\end{trivlist}}
 
\BeforeBeginEnvironment{minted}{\begin{tcolorbox}}%
\AfterEndEnvironment{minted}{\end{tcolorbox}}%
\begin{document}
 
% --------------------------------------------------------------
%                         Start here
% --------------------------------------------------------------
 
\title{Chapter 4 Homework}%replace with the appropriate homework number
\author{Charles Oroko} %if necessary, replace with your course title
\date{8 November 2018}
\maketitle
%Below is an example of the problem environment

% --------------------------------------------------------------

\begin{problem}{4.1}
    If $r0$ initially contains $1$, what will it contain after the third instruction in the sequence below?
    \begin{minted}{gas}
        add     r0, r0, #1          %r0=2
        mov     r1, r0              %r1=2
        add     r0, r1, r0 lsl #1   %r0=6
    \end{minted}

\end{problem}

    %Below is the solution environment
\begin{solution}{}
    $r0=6$
\end{solution}

% --------------------------------------------------------------

\begin{problem}{4.2}
    What will r0 and r1 contain after each of the following instructions? Give your answers in base 10.
    \begin{minted}{gas}
        mov     r0, #1      %r0=1
        mov     r1, #0x20   %r1=32
        orr     r1, r1, r0  %r1=33
        lsl     r1, #0x2    %r1=132
        orr     r1, r1, r0  %r1=133
        eor     r0, r0, r1  %r0=132
        lsr     r1, r0, #3  %r1=1056
    \end{minted}

\end{problem}

\begin{solution}{}
    \begin{align*}
        r0 = & 132 \\
        r1 = & 1056
    \end{align*}
\end{solution}

% --------------------------------------------------------------

\begin{problem}{4.3}
    What is the difference between $lsr$ and $asr$?
\end{problem}

The $lsr$ and $asr$ operations do similar things. They both shifts each bit $n$ bits to the right, losing the least significant $n$ bits.

With the $lsr$ operation, zero is shifted into the $n$ most significant bits.
However, with the $asr$ operation, the $n$ most significant bits become copies of the sign bit (bit $31$).

% --------------------------------------------------------------

\begin{problem}{4.2}
    What will r0 and r1 contain after each of the following instructions? Give your answers in base 10.
    \begin{minted}{gas}
        mov     r0, #1      %r0=1
        mov     r1, #0x20   %r1=32
        orr     r1, r1, r0  %r1=33
        lsl     r1, #0x2    %r1=132
        orr     r1, r1, r0  %r1=133
        eor     r0, r0, r1  %r0=132
        lsr     r1, r0, #3  %r1=1056
    \end{minted}

\end{problem}

\begin{solution}{}
    \begin{align*}
        r0 = & 132 \\
        r1 = & 1056
    \end{align*}
\end{solution}

% --------------------------------------------------------------

\begin{problem}{4.2}
    What will r0 and r1 contain after each of the following instructions? Give your answers in base 10.
    \begin{minted}{gas}
        mov     r0, #1      %r0=1
        mov     r1, #0x20   %r1=32
        orr     r1, r1, r0  %r1=33
        lsl     r1, #0x2    %r1=132
        orr     r1, r1, r0  %r1=133
        eor     r0, r0, r1  %r0=132
        lsr     r1, r0, #3  %r1=1056
    \end{minted}

\end{problem}

\begin{solution}{}
    \begin{align*}
        r0 = & 132 \\
        r1 = & 1056
    \end{align*}
\end{solution}

% --------------------------------------------------------------

\begin{problem}{4.2}
    What will r0 and r1 contain after each of the following instructions? Give your answers in base 10.
    \begin{minted}{gas}
        mov     r0, #1      %r0=1
        mov     r1, #0x20   %r1=32
        orr     r1, r1, r0  %r1=33
        lsl     r1, #0x2    %r1=132
        orr     r1, r1, r0  %r1=133
        eor     r0, r0, r1  %r0=132
        lsr     r1, r0, #3  %r1=1056
    \end{minted}

\end{problem}

\begin{solution}{}
    \begin{align*}
        r0 = & 132 \\
        r1 = & 1056
    \end{align*}
\end{solution}

\end{document}
